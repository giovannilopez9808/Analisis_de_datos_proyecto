\section{Introducción}

Desde inicios del milenio, la población en Latinoamérica ha tenido un crecimiento semejante a una exponencial\cite{CEPAL_2015}. Esto genera problemas de aglomeración urbana, distribución de suelo y movilidad. Las metrópolis de Latinoamérica tienen retos difíciles. Sus problemas se ven reflejados en el tiempo y la distancia de traslado diarios que realiza cada habitante\cite{hall_1978}. El problema de transporte y movilidad urbana es uno de los factores más importantes para las administraciones, siendo un pilar fundamental en el desarrollo social y económico. La aplicación del principio de comodalidad plantea favorecer la promoción e implementación de distintas alternativas que satisfagan las necesidades de transporte, garantizando cobertura, conectividad, flujos continuos, seguridad y eficiencia\cite{pastori_2018}. La congestión del tráfico en las grandes ciudades presenta graves problemas de movilidad. Entre las causas que crean esta congestión se encuentran la falta de planeación y la desarrollo en la infraestructura y la alta densidad poblacional.

Frente a estos problemas en el transporte público, varias metrópolis de Latinoamérica han implementado el principio de comodalidad usando a la bicicleta como medio de transporte alternativo, que dependiendo de la implementación, puede llegar a ser más rápido, cómodo y seguro en comparación a los demás medios de transporte disponibles.
Una correcta implementación de un sistema de bicicletas aporta a la disminución de la congestión del tráfico.

En el mundo existen alrededor de 400 sistemas de bicicletas disponibles al público. Cada sistema tiene particulariades y tecnologías que se ajustam a las necesidades de la región y sus habitantes. La implementación de estos sistemas se debe realizar sobre un estudio que incluye diferentes factores para que se aporte de una manera eficiente hacia la disminución del tráfico.

El área metropolitana de la Ciudad de México tiene un sistema de transporte público que intregra 11 líneas de metro, 7 de autobuses (Metrobus), autobuses no integrados y sistema público de bicicletas. ECOBICI es el sistema público de bicicletas compartidas de la Ciudad de México. El sistema permite a los usuarios registrados tomar una bicicleta de cualquier estación y devolverla a otra más cercana a su destino en trayectos de 45 minutos.

Guadalajara es la segunda metrópolis más importante de México. Su sistema de transporte incluye 2 líneas de tren ligero, una línea de autobuses integrados, autobuses no integrados y el sistema público de bicicletas MiBici.

MiBici es el sistema público de bicicletas de la ciudad de Guadalajara. Es un sistema que está diseñado para realizar recorridos cortos de manera eficiente tomando en cuenta los siguientes puntos:

\begin{itemize}
  \item Instalación de las estaciones en zonas propicias para el sistema.
  \item Delimitación de los polígonos de acción más apropiados.
  \item Estudio de las variables de demanda para el diseño de la red de estaciones.
\end{itemize}